\documentclass[11p]{article}
% Packages
\usepackage{amsmath}
\usepackage{graphicx}
\usepackage{fancyheadings}
\usepackage[swedish]{babel}
\usepackage[
    backend=biber,
    style=authoryear-ibid,
    sorting=ynt
]{biblatex}
\usepackage[utf8]{inputenc}
\usepackage[T1]{fontenc}
%Källor
\addbibresource{references.bib}
\graphicspath{ {./images/} }

% Lite variabler
\def\email{Endo.Axelsson@ga.ntig.se}
\def\foottitle{PMmall}
\def\name{Endo Axelsson}

\title{PMmall \\ \small Gymnasiearbete}
\author{\name}
\date{\today}

\begin{document}

% fixar sidfot
\lfoot{\footnotesize{\name \\ \email}}
\rfoot{\footnotesize{\today}}
\lhead{\sc\footnotesize\foottitle}
\rhead{\nouppercase{\sc\footnotesize\leftmark}}
\pagestyle{fancy}
\renewcommand{\headrulewidth}{0.2pt}
\renewcommand{\footrulewidth}{0.2pt}

% i Sverige har vi normalt inget indrag vid nytt stycke
\setlength{\parindent}{0pt}
% men däremot lite mellanrum
\setlength{\parskip}{10pt}

\maketitle

\section{Metod}
I denna undersökning så kommer en grupp av personer bli indelad i två olika testgrupper: A och B, för att avgöra om musik faktist förbättrar arbetsminnet.
Genre på musik är en stor faktor gällande arbetsminne, så Grupp A får hörlurar där klassiskt musik spelas och grupp B gör undersökningen ljudlöst.
\newline För att avgöra detta så används memory spel som ett verktyg.
I detta fall så ska memory spelet visa 30 styckna kort med olika bilder på, korten visas i 30 sekunder innan dem vänds.
\newline Alla kort ligger på samma ställe för alla, för att det ska vara rättvist och minska slumpäss faktorn.
Målet är att komma ihåg så många kort som möjligt och med hjälp av musiken länka det visuella till ljud.
Arbetet utförs enskilt för att mäta hur mycket varje individ kommer ihåg utan någon påverkan av andra.

\subsection{Memory spel}
Spelen som redan existerade på nätet var för slumpässiga för att få ett rättvist och en stabil resultat.
det fel som inte var passande för undersökningen: korten blandades om för varje omgång som spelades och visades inte i början, vilket gör att de första kortvändingarna ökar felmarginalen.
Så för att undvika detta så ska spelet kodas från grund med hjälp av visual studio code.
Sidan ska laddas upp på glitch därefter.
Glitch är en webbsida som tar in data.
Så från testsidan så skickas svaren in till Glitch och all information är på samma ställe.
Detta underlättar mätningen av resultaten i slutändan.

\subsection{Process}
Testet inleds med att presentera kort för deltagarna vad arbetet kommer att handla om.
Presentationen ska vara kort men effektiv visa varför detta utförs och vad målet med undersökningen är.
Sedan blir deltagarna indelade i två grupper, A och B för att därefter få materialen som krävs för respektiv grupp, hörlurar för de i grupp A.
Innan testet faktist börjar och för att det ska bli så riskfritt så möjligt så kommer instruktionerna precis innan.
Det innehåller spelets innehåll, regler och vad målet är med det.
Länken för memory spelet delas ut i ett classroom där all deltagande har tillgång till.
Väl inne i memory spelet så kommer två alternativ att visa sig: alternativ A och alternativ B.
Detta existerar för att dela in svaren som kommer in.
Grupp A ska välja alternativet A och grupp B ska välja B.
\newline Efter så sätts hörlurarna på för de som har det.
Under en begränsad tid på max 10 minuter så bör alla ha startat spelet.
Det är 30 kort totalt och med hälp av att korten visar sig 30 sekunder i början av spelets gång, så ska deltagarna para ihop så många par så möjligt med så lite fel så möjligt innan tiden tar slut.
Då spelets gång är klar oavsett om det är för att tiden tog slut eller att man parat alla så avslutas det.
När alla som har deltagit är klara så skickas all resultat och data till Glitch.
Vilket gör att det sedan kan göras en beräkning av en slutsats om vilken grupp som presterat bättre än den andra.


\printbibliography

\end{document}
